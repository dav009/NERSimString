On this paper, we presented different implementations of the proposed \texttt{SimString} algorithm. We found out that the implementation of \texttt{SimString} is easily extensible for different approaches. Some of the implementations are more useful for other tasks like fast look-up, and for Named Entity Recognition, which is one of the problematic areas in natural language processing and information extraction. We found out that the \texttt{SuffixTree} and \texttt{HashTable} dictionaries returned the relatively similar and fast average look-up times for potential Named Entity candidates. With the different implementations of the inverted dictionary indexes we have presented, we can conclude that they provide different applications, and they were useful in evaluating the performance of the implementation for comparison purposes.\\

Some of the research areas that can be looked further in detail, for example, would be building a \texttt{bootstrapper} for learning Named Entities based on the different implementations of the \texttt{SimString} algorithm, such as the \texttt{NERSimString} dictionaries we presented on this paper. Adding extra similarity measures alternatives for testing and training might also help finding out the best approach for that kind of applications.\\

More sophisticated NER systems can benefit from the \texttt{SimString} implementations by further expanding it with \textit{signature information} which can be extracted from the dictionaries. This info could include anything from finding the \textit{upper and lower boundaries} of NE candidates to applying Noun-Phrase Chunking on the NE Candidates returned from the dictionaries.\\

Besides the potential \textit{bootstrapping} oriented implementation or more sophisticated NER system; adding more support for \texttt{Memory-Mapped HashTable} dictionaries might also be useful for other applications. For the basic NER system, it is expected to see that the \texttt{Mapped HashTable} Dictionaries outputted the slowest search look-up times for Named Entity candidates. However, this does not mean that the \texttt{Mapped HashTable} dictionaries are not a good candidate implementation for other look-up applications, that require a lot of data to be held on the memory. For example, for spell-checker application dictionaries, especially for languages with a large lexicon \textit{(e.g., highly-inflected languages)}, the use of \texttt{Mapped HashTable} dictionary implementation would be very beneficial.\\

Another functionality could be easily added in order to improve this algorithm when searching for Name Entities,
that is for example saving extra information, (a small simstring dictionary for example) with signatures for allowing a better candidate selection with dealing with this task.
When it comes to this task, the simstring structure could also be extended for holding more than one value per key,this  allowing a single dictionary to hold different kinds of name entities.\\

As a final note, for this project, we would like note that researching about different implementation techniques and learning about the differences in application to search look-up was an enjoyable learning task. Because information extracting and search techniques is one of the most challenging--and at the same time rewarding tasks in NLP, we plan on continuing on working on the mentioned research areas for future research.