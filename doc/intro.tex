
Due to the problematic identification of Named Entities, collecting Named Entities in the field of Information Extraction and Natural Language Processing has been a popular task during the past years. There have been more than several methods that deal with this specific task; both from a statistical point of view, and rule-based method point of view, for extraction of Name Entities(NE) from raw text. On this paper, we present \texttt{NERSimString}, which is a Java library, based on an approximate dictionary matching algorithm\cite{Okazaki:Coling2010}\footnote{http://www.chokkan.org/software/simstring}that uses various similarity measures. This algorithm targets the retrieval of strings faster for Named Entity Recognition systems. The main idea behind the \texttt{NERSimString} is that it creates a library for storing Name Entity dictionaries, and at the same time provides a quick Named Entity annotation tool for the given named entity dictionaries generated by the system.\\

In most NER systems, creating  and storing NE dictionaries can raise some problems. Even in most state-of-the-art NER systems, generated NE dictionaries, to be used as fixed-lists later, impose the following issues that still remain challenging to resolve:
\begin{itemize}
	\item Even in a restricted domain, the number of Named Entities in a dictionary can be overwhelming.
	\item Considering the size of a corpus, when the fixed-dictionaries are given, the final goal of finding all the NEs contained in a corpus can still be a hard task.  
	\item Even in semi-supervised methods that are trained to recognize the NEs require fast searching tools; because the training requires searching and finding the location of a given set of NE that is located in a corpus.
\end{itemize}

Therefore, in order to make the annotation task of Named Entities a tractable problem, a method that looks for words quickly in a huge dictionary becomes mandatory. The \texttt{NERSimString} algorithm precisely targets to handle this specific problem and aims to achieve an accurate and fast searching method.
